\usepackage[a4paper, textheight=22cm, right=3cm, heightrounded]{geometry}
\usepackage[utf8]{inputenc} 	%les accents français
\usepackage[T1]{fontenc}        %encodage clavier français
\usepackage[inline]{asymptote}  
\usepackage[english]{babel}   	%typo et césures françaises
\usepackage{amsmath, amsfonts, amsthm, amssymb}	% les maths
\usepackage{enumerate}		%listes enrichies
\usepackage{graphicx}           %insertion de figures
\usepackage{xcolor}
\usepackage{color,colortbl}
\usepackage{listingsutf8}       %intégration utf8 dans listing
\usepackage{calc}
\usepackage{multicol}
\usepackage{tabularx}           %pour des tableaux sur mesures
\usepackage{tabulary}
\usepackage{import}             %pour une seule compilation à partir de
                                %l'ensemble des sources
%\usepackage{frcursive}         %une fonte cursive
\usepackage[ %captio et sa personnalisation
labelfont={small,bf},textfont=it,figurename=Fig,
%le compteur fig repart à 0 à chaque nouveau chapitre
%lorsque la légende ne contient qu'une seule ligne
%elle doit tenir compte de la justification
figurewithin=section,singlelinecheck=false,
%Alignement à gauche
justification=raggedright]{caption}

%%%% Des figures sans label juste un numéro%%%%%%%%
%On modifie le label de la légende
%\DeclareCaptionFormat{Fig}{\begin{cursive}Fig #2\end{cursive}} 
%\captionsetup{captionformat=Fig,labelsep=quad}
%On change la couleur du label
%\DeclareCaptionFont{gray}{\color{gray}}
%\captionsetup{labelfont={gray,bf}}
%%%%%%%%%%%%%%%%%%%%%%%%%%%%%%%%%%%%%%%%%%%%%%%%%%%
\usepackage{floatrow} %floatrow et sa personnalisation
\floatsetup{style=plain,capposition=TOP} 
\usepackage[Lenny]{fncychap} %Mise en page des chapitres

%Définition d'un nouveau langage(asy) pour listing
\definecolor{bckgcolor}{rgb}{0.9,0.9,0.8}
\definecolor{keywordstc}{rgb}{0.7,0,0}
\definecolor{comments}{rgb}{0,0,1}
\lstdefinelanguage{asy}
                  {morekeywords={arrowbar, path, draw, label, real, pair,
                      picture, ticks, pen, bool, void},
                    sensitive=false,
                    morecomment=[l][\color{comments}]{//},
                    morecomment=[s]{/*}{*/},
                    morestring=[b]"
                  }
 \lstset{
   frameround=tttt,
   frame=single,
   numbers=left,
   language=asy, 
   numberstyle=\tiny,stepnumber=1, 
   basicstyle=\scriptsize\ttfamily,
   keywordstyle=\color{keywordstc}, 
   numbersep=5pt,
   tabsize=2,
   backgroundcolor=\color{bckgcolor}
}
%%%%%%%%%%%%%%%%%%%%%%%%%%%%%%%%%%%%%
% Majuscules droites en mode math %%%
%%%%%%%%%%%%%%%%%%%%%%%%%%%%%%%%%%%%%
\DeclareMathVersion{majDroites}
\DeclareSymbolFont{majuscules}{OML}{cmm}{m}{it}
\SetSymbolFont{majuscules}{majDroites}{OT1}{cmr}{m }{n}
\DeclareMathSymbol{A}{\mathalpha}{majuscules}{"41}
\DeclareMathSymbol{B}{\mathalpha}{majuscules}{"42}
\DeclareMathSymbol{C}{\mathalpha}{majuscules}{"43}
\DeclareMathSymbol{D}{\mathalpha}{majuscules}{"44}
\DeclareMathSymbol{E}{\mathalpha}{majuscules}{"45}
\DeclareMathSymbol{F}{\mathalpha}{majuscules}{"46}
\DeclareMathSymbol{G}{\mathalpha}{majuscules}{"47}
\DeclareMathSymbol{H}{\mathalpha}{majuscules}{"48}
\DeclareMathSymbol{I}{\mathalpha}{majuscules}{"49}
\DeclareMathSymbol{J}{\mathalpha}{majuscules}{"4A}
\DeclareMathSymbol{K}{\mathalpha}{majuscules}{"4B}
\DeclareMathSymbol{L}{\mathalpha}{majuscules}{"4C}
\DeclareMathSymbol{M}{\mathalpha}{majuscules}{"4D}
\DeclareMathSymbol{N}{\mathalpha}{majuscules}{"4E}
\DeclareMathSymbol{O}{\mathalpha}{majuscules}{"4F}
\DeclareMathSymbol{P}{\mathalpha}{majuscules}{"50}
\DeclareMathSymbol{Q}{\mathalpha}{majuscules}{"51}
\DeclareMathSymbol{R}{\mathalpha}{majuscules}{"52}
\DeclareMathSymbol{S}{\mathalpha}{majuscules}{"53}
\DeclareMathSymbol{T}{\mathalpha}{majuscules}{"54}
\DeclareMathSymbol{U}{\mathalpha}{majuscules}{"55}
\DeclareMathSymbol{V}{\mathalpha}{majuscules}{"56}
\DeclareMathSymbol{W}{\mathalpha}{majuscules}{"57}
\DeclareMathSymbol{X}{\mathalpha}{majuscules}{"58}
\DeclareMathSymbol{Y}{\mathalpha}{majuscules}{"59}
\DeclareMathSymbol{Z}{\mathalpha}{majuscules}{"5A}
\mathversion{majDroites} 

