%% Version de la macro figAsy utilisant minipage et listings
%%%%                  debut macro                       %%%%
\newcommand{\figAsy}[5]
{\noindent
\begin{minipage}[c]{0.4\textwidth}
\begin{figure}[H]
\ffigbox[\FBwidth]
{\includegraphics[width=#1, height=#2]{#3.pdf}}
{\caption{\label{#5}\footnotesize #4}}
\end{figure}
\end{minipage}
\hfill
\begin{minipage}[c]{0.55\textwidth}
 \tiny
   \lstinputlisting[inputencoding=utf8/latin1,tabsize=7]{#3.asy}
 \normalsize
\end{minipage}
\vspace{0.5cm}
}
%% La même macro mais avec la possibilité d'indiquer la plage de ligne
%% à afficher.
\newcommand{\figAsyLine}[6]
{
\begin{minipage}[c]{0.45\textwidth}
\begin{figure}[H]
\ffigbox[\FBwidth]
{\includegraphics[width=#1, height=#2]{#3.pdf}}
{\caption{\footnotesize #4}}
\end{figure}
\end{minipage}
\hfill
\begin{minipage}[c]{0.55\textwidth}
 \tiny
   \lstinputlisting[inputencoding=utf8/latin1,firstline=#5,lastline=#6]{#3.asy} 
 \normalsize
\end{minipage}
\vspace{0.5cm}
}

%% La macro pour insérer uniquement le code%%%%%%%%%%%%%%%%%%%%%%
\newcommand{\code}[2]
{
\begin{center}
\begin{minipage}{#1}
\lstinputlisting[inputencoding=utf8/latin1]{#2.asy}  
\end{minipage}
\end{center}
}

%%Définition de la numérotation des subsection%%%%%%%%%%%
\renewcommand{\thesubsubsection}{\alph{subsubsection}.}
%%%%                   fin macro                         %%%%
